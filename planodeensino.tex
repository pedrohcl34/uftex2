\documentclass[12pt,consuni]{uftex2}
%-------------------------------------------------------------------------------------------------------------------------%
% Endereço Institucional
%-------------------------------------------------------------------------------------------------------------------------%
\address{109 Norte, Avenida NS 15, ALCNO 14, Bloco Bala 2 - Sala 21}
\cep{77402-970}
\phone{(63) 3232-8027}
\mail{comppalmas@uft.edu.br}
\department{AL}
\city{Palmas}
%-------------------------------------------------------------------------------------------------------------------------%
% Dados da Disciplina
%-------------------------------------------------------------------------------------------------------------------------%
\term{2016}{2}
\title{Sistemas Digitais}
\carga{60}
\codigo{CET191}
\matricula{2143480}
\turma{T.SD}
\creditos{4}
\author{Tiago da Silva}{Almeida}
\tipo{Obrigatória}
%-------------------------------------------------------------------------------------------------------------------------%
\begin{document}

\maketitle

\chapter{Ementa}

Álgebra Booleana; Portas Lógicas; Circuitos Combinacionais; Projeto de sistemas combinacionais; Circuitos Seqüenciais; Flip-Flops; Contadores e Registradores, máquinas de estado finitos; Projeto de Sistemas Seqüenciais ; Aritmética Digital: circuitos e Operações aritméticas; Interface com o Mundo Analógico; Dispositivos de Memória; Dispositivos de Lógica Programável.

\chapter{Objeticvos}

\section{Geral}

\begin{itemize}
\item Apresentar os conceitos de lógica digital, de maneira a proporcionar uma visão interna dos circuitos que compõe um computador.
\end{itemize}

\section{Específicos:}

\begin{itemize}
\item O estudo de Sistemas Digitais possibilita a abstração de conceitos, dificilmente visíveis pelo emprego de ferramentas, visto que serão estudados os cálculos binários executados pelos componentes de um computador através da Álgebra de Boole, portas lógicas e demais teorias. 

\item O conhecimento das bases de um componente computacional possibilitará ao aluno desenvolver com maior clareza as aplicações, bem como projetar sistemas que envolvam de alguma forma a necessidade de conhecimento do funcionamento de um computador. 
\item Os conhecimentos adquiridos nesta disciplina fornecerão ao aluno os subsídios teóricos e lógicos para explicar os circuitos básicos de um computador.
\end{itemize}


\chapter{Conteúdo Programático}

\conteudo{Álgebra das variáveis lógicas}
\conteudo{Representação de variáveis lógicas por tensões elétricas}
\conteudo{Teoremo de D Morgan e diagramas de Venn}
\conteudo{Funções Lógicas}
\conteudo{Mapas de Karnaugh e simplificação de funções lógicas}
\conteudo{Circuitos combinacionais básicos}
\conteudo{Codificadores e decodificadores}
\conteudo{Multiplexadores e demultiplexadores}
\conteudo{Latchs, Flip-flop mestre-escravo, JK, D} 
\conteudo{Registradores, transferência entre registradores e regisitradores de deslocamento}
\conteudo{Contadores em anel, anel torcido, sincrônos e por pulsação}
\conteudo{Representação de números com sinal}
\conteudo{Somadores, subtratores, multiplicadores e divisores}
\conteudo{Memória de acesso aleatório, somente leitura. Memórias programáveis e apagáveis}
\conteudo{Circuitos sequenciais de Mealy e Moore}
\conteudo{Controladores de transferência de registradores, sinsíveis à comandos múltiplos e de registrador de deslocamento}
\conteudo{Instruções de computadores e microprogramação}
\conteudo{Instruções de um, dois e três bytes no 8080, para movimento de dados, aritméticas e de I/O}
\imprimirConteudo

\chapter{Metodologia}

\section{Ensino}

\begin{itemize}[partopsep=0pt, topsep=-\parskip, parsep=0pt, itemsep=0pt, leftmargin=.5cm]
\item Aulas expositivas;
\item Lista de Exercícios;
\item Avaliações.
\end{itemize}


\begin{itemize}
\item Experimentos em laboratório;
\end{itemize}


\begin{itemize}
\item Lógica Matemática;
\item Organização de computadores;
\item Compiladores.
\end{itemize}


\begin{itemize}
\item Simulador de linguagem Assembly;
\item Laboratório de Hardware;
\item Laboratorista
\end{itemize}

\section{Avaliação}

Avaliações valendo de 0 a 10 pontos com peso de 70\%. Trabalhos e exercícios em sala de aula valendo de 0 a 10 pontos com peso de 30\%. A média final é obtida perante o seguinte cálculo:

\[
M_f= \left( \left( \frac{ \sum_{i=1}^{n} P_i }{n} \right) \times 0,7 \right)+
     \left( \left( \frac{ \sum_{j=1}^{m} T_j }{m} \right) \times 0,3 \right)
\]
tal que, $Pi$ corresponde a nota da i-ésima avaliação, $Tj$ corresponde a nota do j-ésimo trabalho ou exercícios, $n$ o número de avaliações e $m$ o número de trabalhos.

Sendo que os alunos aprovados deverão obter obrigatoriamente $M_f \geq 7,0$. Alunos com media final $4,0 \leq M_f \geq 6,9$ farão exame final. Por fim, alunos com média final $M_f<4,0$ estarão reprovados. O aluno em exame final deverá atingir nota $\frac{M_f+Exame}{2}\geq 5,0$.

Em caso algum aluno perca alguma das $P_i$ avaliações, por algum motivo especial, como em caso de doença e devidamente protocolado o pedido de substitutiva, será aplicada somente uma avaliação substitutiva ao final do semestre englobando todo o conteúdo ministrado.

\chapter{Bibliografia}

\section{Básica}
\begin{enumerate}
\item TOCCI, Ronald J., WIDMER, Neal S. Sistemas Digitais Princípios e Aplicações. Editora LTC – RJ, 11 ed. 2005.
{\color{red} \item CAPUANO, Francisco G. Sistemas Digitais - Circuitos Combinacionais e Sequenciais. Erica - SP, 144p, 2014.
\item VAHID, Frank. Sistemas Digitais - Projeto, Otimização e Hdls. Bookman - SP, 558 p., 2008.
\item DIAS, Morgado. Sistemas Digitais. Princípios E Prática. FCA - RJ, 2 ed., 506 p., 2011.
\item SZAJNBERG, Mordka. Eletrônica Digital - Teoria, Componentes e Aplicações. LTC - RJ, 476 p., 2014.
\item HETEM JUNIOR, Annibal. Fundamentos de Informática - Eletrônica Básica para Computação. LTC - RJ, 234 p., 2009.}
\end{enumerate}

\section{Complementar}

\begin{enumerate}
\item TOCCI, Ronald J., AMBROSIO, Frank J., LASKOWSK, Lester P. Microprocessors and microcomputers :hardware and software. Prentice Hall,5 ed., 565p., 2000.
\item BREY, Barry B. The Intel microprocessors :8086 8088, 80186 80188, 80286, 80386, 80486, Pentium, Pentium Pro, and Pentium II processors : architecture, programming, and interfacing. Prentice Hall, 5 ed., 966 p., 2000.
\item RABAEY, Jan M. Digital integrated circuits :a design perspective. Prentice Hall, 702 p., 1996.
\item IDOETA, Ivan Valeije. Elementos de eletrônica digital. Érica - SP, 4 ed., 524 p., 2010.
\item GARCIA, Paulo Alves., MARTINI, Jose Sidnei Colombo,.Eletronica digital:teoria e laboratorio. Erica - SP, 2.ed., 182p., 2008.
\item GARUE, Sergio. Eletronica digital:circuitos e tecnologias LSI e VLSI. Hemus - SP, 299p., 2004.
\end{enumerate}


\end{document}

